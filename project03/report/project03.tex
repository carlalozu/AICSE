\documentclass[unicode,11pt,a4paper,oneside,numbers=endperiod,openany]{scrartcl}
\usepackage{amsmath}
\usepackage{amssymb}
\input{assignment.sty}

\begin{document}


\setassignment
\setduedate{Wednesday 26 June 2024, 23:59}

\serieheader{AI in the Sciences and Engineering}{Spring Semester 2024}
            {Student: Carla Judith L\'opez Zurita}
            {}{Project 3}{}
\newline

The main objective of the project is to apply the concepts learned in class
related to Neural Differential Equations and Hybrid Workflows.


\section{Inverted pendulum}\label{sec:task1}
The objective of this task is to control the inverted pendulum problem using a
Neural Network. The inverted pendulum is a classic problem in control theory,
where the goal is to balance a pendulum in an upright position by applying an 
external force $F$ to the cart that holds the pendulum. 
The system is described by the following differential equation:
\begin{equation}
    (M+m) \ddot{x} + m l \ddot{\theta} \cos(\theta) - m l \dot{\theta}^2 \sin(\theta) = F
    l \ddot{\theta} + g \sin(\theta) = 0
\end{equation}
where $\theta$ is the angle of the pendulum with respect to the vertical,
$g=9.81$ m/s$^2$ is
the acceleration due to gravity, $l=1.0$ m is the length of the pendulum,
$m=0.1$ kg is the
mass of the pendulum, $M=1.0$ kg is the mass of the cart and $F$ is the force
applied to the cart in Newtons.

\subsection*{Solving the coupled ODE system}
First, we need to solve the coupled ODE system to simulate the dynamics of the
inverted pendulum. I used Runge-Kutta 4th order method to solve the system.
We simulated the system with a sinusoidal force $F(t) = 10 \sin(t)$.
The results are shown in Figure~\ref{fig:ode}.
\begin{figure}[h]
    \centering
    \includegraphics[width=0.8\textwidth]{figures/ode.png}
    \caption{Simulation of the inverted pendulum with a sinusoidal force.}
    \label{fig:ode}
\end{figure}

\subsection*{Learning to balance the pendulum}

\section{Task 2}\label{sec:task2}


\section{Bonus Task}\label{sec:task3}


\bibliographystyle{plain}
\bibliography{bibliography}

\end{document}
